\documentclass{article}
\usepackage{a4wide}
\usepackage{cite}



%% if your are not using LaTeX2e use instead
%% \documentstyle[bnaic]{article}

%% begin document with title, author and affiliations

\title{Seminarium TI 3700\\ Data Analysis on Twitter: An Overview (working title)}
\author{A. Hambenne  \and
    O. Maas \and
    R. Oosterloo}
\date{}

\pagestyle{empty}

\begin{document}
\maketitle
\thispagestyle{empty}

\begin{abstract}
This is the abstract of my paper.
This is the abstract of my paper.
This is the abstract of my paper.
This is the abstract of my paper.
This is the abstract of my paper.
This is the abstract of my paper.
This is the abstract of my paper.
This is the abstract of my paper.
This is the abstract of my paper.
This is the abstract of my paper.
This is the abstract of my paper.
This is the abstract of my paper.
\end{abstract}


\section{Introduction}

\newpage

\section{Data}

\subsection{Twitter API}

When Twitter was conceived, the creators chose to adopt an API (Application Specific Interface) for third party developers. This has proven to be very
beneficial to both Twitter's popularity and alot of its users, including researchers.

An overview of the API structure and some useful fields:
\begin{enumerate}
\item \textbf{Tweets} The short messages consisting of 140 characters sent by users
	\subitem \textbf{Coordinates} Returns the longitude and latitude position from where a specified tweet has been sent
	\subitem \textbf{Entities} Returns the entities (url, hashtag, etc.) that are embedded into the tweet
	\subitem \textbf{Lang} Returns the language in which a tweet has been sent
	\subitem \textbf{User} Returns the user that sent a tweet
\item \textbf{Users} The users that send the tweets
	\subitem \textbf{Description} Returns the description that the user defined for his account
	\subitem \textbf{Location} Returns the location that the user defined for his account
	\subitem \textbf{Status} Returns, if possible, the user's latest tweet
\item \textbf{Entities} Images, hashtags and other entities that tweets can hold
	\subitem \textbf{Hashtags} Returns the hashtags that have been parset out of a tweet
	\subitem \textbf{Media} Returns the media that is associated with a tweet
	\subitem \textbf{urls} Returns the urls embedded in a tweet
	\subitem \textbf{user\_mentions} Returns the users that have been mentioned in a tweet
\item \textbf{Places} Predefined places on Twitter that users can associate with
	\subitem \textbf{Country} Returns the country associated with a place
	\subitem \textbf{Full\_name} Returns the full name of a place, for example Kanses ``City, Kansas''
	\subitem \textbf{Place\_type} Returns the type of a place, for example ``city''
\end{enumerate}

When the above fields are combined, it is possible to create datasets of Tweets that fulfill a certain specification. 
These sets can then be used to extract specific information. However, defining your own dataset may introduce a certain bias, 
and sometimes you need a random sample of tweets to analyse. For this purpose, Twitter has a streaming function that
third party developers can tap into to retrieve random tweets. Unfortunately, the Twitter stream can only retrieve tweets that are less than a few days old.

The following of this article describes algorithms that dive into the big data of Twitter, how to extract useful information from the tweets, and what the information
can be applied to.


\section{Single tweet analysis}
In this section we will discuss the different types of analysis that can be done on one single tweet. This usually is also the basis for a analysis of a larger amount of tweets or even on a stream of tweets. 
\subsection{Data enrichment}
At the very basis of the analysis of a tweet is data enrichment. Because the length of a tweet is restricted to 140 characters, the users of Twitter use language in creative ways. This way of communicating makes it sometimes, even for humans, hard to understand what a tweet is about. To get a better understanding of the meaning of a tweet, a context is needed. For humans this is done by linking the concepts from the tweet to earlier acquired knowledge. The same thing can be done by computers. Two stages can be defined in this process, first the extraction of the concepts from the tweet, after that a context can be created by linking the tweet to other sources.
\subsubsection{Concept extraction}
In this case by concepts we mean named entities such as names of locations, people and organizations. To recognize these entities multiple learning algorithms or even combinations of learning algorithms have been proposed. The most common is statistical learning. Think of algorithms like Maximum Entropy Models, Hidden Markov Models or neural networks. Other proposed algorithms are AdaBoost.MH, employed memory-based learning, transformation-based learning, support vector machines and conditional random fields. All of them in decreasing frequency of use.\cite{EntityRecognition} Florian et al. have shown that combining multiple classifiers can increase the performance, compared to the best performing classifier, by 17-21 \%.\cite{ClassifierCombination} After the entities have been found, labels are assigned to distinguish between a person, an organization or location. Then Abel et al. propose to use the RDF.\cite{AdaptiveSearch} By doing so sources from the Semantic Web (e.g. DBpedia \footnote{http://dbpedia.
org/}) can now be leveraged to enrich the data. Another advantage of using RDF is that RDF is a Semantic Web standard recommended by the W3C\footnote{http://www.w3.org/RDF/}. Because W3C recommends to use this model, most likely a lot of other sources also use this model. This will give a much bigger knowledge base to enrich the data from.
\subsubsection{Context creation}
The problem of linking tweets to another source of information poses the biggest issue in data enrichment. The tweets themselves do not provide much information. Fortunately, Kwak et al. have shown that the topic of over 85\% of the of the tweets is related to news \cite{newsmedia}. This knowledge can be leveraged to create a context for most of the tweets.  This is what Abel et al. proposed to do\cite{enrichmentForProfiling}. They linked tweets to mainstream news networks like CNN, BBC and New York Times.

Linking to just news articles narrows the problem significantly, however, actually linking the tweets and the new articles still is an issue. Abel et al. propose two main linking strategies, divided into five methods total.
\begin{enumerate}
\item \textbf{URL-based strategy}\\
This strategy is based upon the observation that many tweets contains URLs. Usually these URLs refer to news articles the tweets are closely related to. Two URL-based strategies are proposed
	\subitem \textbf{Strict URL-based} defines an article $a$ and a tweet $t$ to be related when $t$ contains an URL to $a$.
	\subitem \textbf{Lenient URL-based}, this method is an extension of the strict URL-based method. This extension entails that if a tweet $t$ has a link to an article $a$, all tweets that are retweeting $t$ are considered linked to $a$. The same counts for tweets that are in responds to $t$, these tweets will also be considered linked to $a$.
\item \textbf{Content-based strategy}\\
Content-based strategies rely on the idea of finding as much similarity as possible between the content of a tweet and the content of a news article. The most similar tweet-article pair is considered to be related and therefor they are linked.
	\subitem \textbf{Bag of words}, in this method a tweet is represented as a vector of word frequencies. For a news article we can also create a vector the represent it. To link an article to a tweet we need to calculate the similarity between the tweet and the article and pick the one with the highest similarity. The similarity is calculated by the $TF \times IDF$ (Term Frequency times Inverse Document Frequency): the number of times a word from the tweet occurs in a news article, multiplied by the logarithm of the total number of articles over the number of articles the word occurs in. \cite{tfidf}
	\subitem \textbf{Hashtag-based} is based upon the same concept as the Bag of words method. The difference is that we do not consider the words in the article, instead we only use the hashtags in the representation of a tweet to calculate the $TF \times IDF$ with.
	\subitem \textbf{Entity-based} is also based upon the Bag of words method. The entity-based method does not consider words in an article. It relies on the entities that are mentioned in the article to make a vector to represent the article. 
\end{enumerate}
Using these method a broader context can be created. This still is not much more useful than the tweet on its own. For the article to be actually useful the topics of the articles have to be extracted. To do this Natural Language Processing and machine learning can be used to extract the entities from the article. OpenCalais\footnote{www.opencalais.com} is a web service that does just that. It returns not only the entities, it will also return facts and events hidden in the text, which will make the result much more useful. 

\subsection{Sentiment analysis}
Sentiment analysis or opinion mining is "a linguistic analysis technique where a body of text is examined to characterize the tonality of the document."\footnote{http://lexicon.ft.com/Term?term=sentiment-analysis} In other words, the goal of a sentiment analysis is to extract the writers opinion on some topic from a body of text. This body of text can very well be a tweet, and that combined with the popularity of microblogging amongst internet users, makes Twitter a very interesting body of knowledge about the opinion of a large group of people. But in order to get to that knowledge the data has to be analyzed fist. 
The most important and most obvious issue is how to extract an opinion from a tweet. In the next part we will discuss a few ways to extract an opinion from a tweet.
\subsubsection{Semantics }
The most important way to extract a sentiment from a tweet is by analyzing its semantics. After all, tweets are written text. A sentiment can be assigned to a tweet using machine learning algorithms. Pang et al. \cite{machineLearning} have done a comparative study of three standard machine learning algorithms for sentiment analysis: 
\begin{enumerate}
\item \textbf{Naive Bayes} is a fair simple classifier that operates under the assumption that the presents or absence of features does not relate to the presents of other features. If we apply that on Twitter we would get $c = \arg\max_{c\in C} P(c|w)$ where $c$ is a sentiment class, $C$ the collection of sentiment classes and $w$ is a tweet. \cite{sentAnalysis}
\item \textbf{Maximum Entropy} is a model that, unlike Bayes, does not make an independence assumption. It operates under the philosophy that "we should choose the model making the fewest assumptions about the data while still remaining consistent with it".\cite{machineLearning} \\
The formula is of the form $P(c|d) = \frac{1}{Z(d)}\exp(\Sigma_i \lambda_{i,c}F_{i,c}(d,c))$, where $Z(d)$ is a normalization function, $\lambda_{i,c}$ a weighted vector determining the significance of the feature and $F_{i,c}$ is a class/feature function. 
\item \textbf{Support Vector Machine} is fundamentally different from Naive Bayes and Max Entropy because it is not a probabilistic model. The basic idea behind a support vector machine is that a vector is trained to separate the sentiments. Not only will it separate the training data, it will do so at the largest possible margin. 
\end{enumerate}
It turns out that over the whole the Support Vector Machine performs the best on the sentiment analysis\cite{machineLearning}. The differences, however, are not substantially large. 

\subsubsection{Emoticons}
Other than semantical analysis we can also use other data from tweets. A good sentiment indicator are emoticons. Twitter users use emoticons (e.g. ":-)", ":(" or "=(") to indicate their mood.\cite{twitterSentiment} Since the number of emoticons is limited a sentiment can easily be assigned manually. The information from the emoticons can be leveraged in two ways
\begin{enumerate}
\item \textbf{Labeling:} The emoticons can be used to assign a sentiment to a tweet. This can be a very useful asset to take into account when assigning a sentiment to a tweet. The emoticons can be considered unigrams feature for some classifier. There they will be strong evidence for a certain sentiment.
\item \textbf{Training:} Because of their clear evidence for a sentiment emoticons can also be used to build a training corpus for a supervised learning algorithm. Twitter can be queried for "positive" emoticons and "negative" emoticons and then, by assuming the emoticon reflects the sentiment of message, the tweet can be labeled accordingly.\cite{moodLens} 
\end{enumerate}
We now see that emoticons can be used to extract a sentiment from a tweet. However, they are not sufficient to just use emoticons to label the tweets. Simply because not all tweets contain emoticons.
\subsubsection{Usage}
The possibilities of sentiment analysis are not too hard to imagine. Every company wants to know how their newly launched product is doing, film makes want to know what people think about their new movie and politicians want to know how they are doing for the next elections. Using sentiment analysis all this information will become available\cite{machineLearning}. 
\subsection{Metadata analysis (opt)}

\section{Tweet stream analysis}
\subsection{Statistics}

\subsection{User Classification}
User classification is the problem of classification of users based on (1) their metadata such as age, location and job and (2) the tweets a user has posted. In this chapter we will discuss User Models, some features used to classify users and some use-cases. 
\subsubsection{User Models}
Before it is possible to classify a user, it is useful to create a model of a user. In their paper Fabel et al compare the following three ways to model a user. \cite{usermodel}
\begin{itemize}
\item Hashtag-based
\item Entity-based
\item Topic-based

\end{itemize}
After all these approaches are used on the same dataset it appears that the variety of entity-based profiles was much higher than the hashtag based approach. \cite{usermodel} Furthermore the hashtag-based approach fails to create a profile for 5.5\% of the users where the other approaches don't have this problem.  

\subsubsection{Classification Features}
To classify users a set of features is needed to distinguish between users. In this section we will present a list of frequently used features to classify users.  
\\\\
\textbf{Linguistics}\\
Linguistics can say a lot about the interests of a user. To classify a user Prototipical Words(PW's) can be used.  \cite{userclasst} PW's are words that are more used by users in class $A$ than users in class $B$. In a twitter-context also Prototipical Hashtags can be used.
\\\\
\textbf{Interaction Profiles}\\
Users interaction profiles is a profile created by \cite{pol} to capture interactions between users and other accounts. An interaction is a retweet or a reply. \cite{usermachine} also uses friend accounts as a measurement in interaction between users. A friend account of user $x$ is an account that is followed by user $x$. 
\\\\
\textbf{Tweeting Behavior}\\
Tweeting behavior is a set of characteristics which shows us the way a users interacts with twitter. Things like the average number of messages, number of replies \& number of friends seems a good quantifier for classification. According to \cite{usermachine} these features are useful but not really discriminative. 
\\\\
\textbf{Profile}\\
Profile information can be helpful to classify a user, but it should be used with care due to the amount of fake, inconsistent or incorrect information. According to \cite{usermachine} 80\% of the users provide a location and 48\% of the users provide a short bio. Furthermore \cite{usermachine} was able to extract a gender for 80\% of the users, but with a very low accuracy. \cite{usermachine} also researched the profile picture/avatar of twitter users. They found out that these pictures can often be misleading as 20\% of the found profile pictures was not the owner but another person such as a celebrity. 

\subsubsection{Usecase: Predicting Political Preferences}
During election time it can be useful to predict the political preferences of a user. In \cite{pol} a solution to this problem is presented. As not all tweets from users reflect their political preference a preprocessing of data is done in order to filter the tweets. 

To achieve this a combination of (political) hashtags in combination with twitter activities is used. To know if a user mentions an account of an election candidate \cite{pol} semi-automatically scraped a number of these accounts. Furthermore they collected 181972 tweets about politics, before processing these tweets non-personal and spam accounts had to be filtered.

To classify the users a one vs all classifier is trained. (Which will recorgnize one party vs all other parties.) After this all information is combined to predict a political party for each user. 
\subsection{Tag based analysis}

Twitter allows their users to contribute their own social annotations to their tweets in the form of tags. These tags can be arbitrarily generated by users 
which make them hold interesting information about both the tweet and the user that posted it.

\begin{enumerate}
\item \textbf{Linear-Weighted Hybrid}\\

%TODO: Technische stuff begrijpen zodat ik die hier neer kan zetten + bibliography laten doorverwijzen

\item \textbf{Pair-Wise Interaction Tensor Factorization} The Pair-Wise Interaction Tensor Factorisation is a state-of-the-art algorithm that has
been developed for the purpose of tag-based recommendation systems. The model can be adapted to compute recommendations from tags used by Twitter.
	
\end{enumerate}


\subsection{Event identification}

Event identification is the problem of identifying events(e.g. concerts, festivals \& incidents) in a twitter-stream. The events identified range from Worldwide to local events. Twitter-messages related to an event can be a useful source of information as they often spread news earlier than the 'old' media (e.g News Papers \& News Sites)

To identify an event using an algorithm, a definition of an event is needed. \cite{eventident} defines an event as: "a real world occurence $e$ with (1) an associated Time Period $T_e$ and (2) a time-ordered stream of Twitter messages $M_e$, of substantial volume, discussing the
occurrence and published during time $T_e$."

\subsubsection{Event Identification Issues}
In this section we will discuss several issues regarding Event Identification and the solutions associated with them.
\\\\ 
\textbf{Separation of Event and Non-event Content}\\
Because not all tweets on twitter are event content, a seperation of Event and Non-event content is done prior of identifying events. To achieve this separation, \cite{eventident} uses an incremental clustering algorithm(ICA). ICA doesn't require an number of clusters before the execution of the algorithm. This is an useful feature in the constantly changing twitter-environment.

After the clustering a number of features is used to distinguish between event and non-event clusters. The following features are used by \cite{eventident} to identify an event cluster. 
\begin{itemize}
  \item Temporal: The characteristics of the volume over the time $T_e$ associated with an event. 
  \item Social: The interactions between users(e.g. retweets, replies \& mentions) during events could differ from other activities. 
  \item Topical: Tweets about events tend to center more around one central topic. Furthermore messages in event clusters are more likely to share one or more event-specific key-terms.  
  \item Twitter-centric: Some activities behave like an event but are Twitter-centric.(e.g. Tweets like 'Goodmorning!') These activities need to be filtered. 
\end{itemize}

With this list of features two event-classifiers are trained using two different techniques. The first classifier is trained using the "WEKA tookit"\footnote{http://www.cs.waikato.ac.nz/ml/weka/}, in the second classifier the  Naive-Bayes technique is used. According to \cite{eventident} the first classifier outperforms the second classifier in the separation of event and non-event clusters. 
\\\\
\textbf{Aligning Tweets with an Event} \\
On average thousands of tweets are sent every minute using Twitter(bron) . One of the main challenges in Event Identification is the alignment of tweets with their respective event. \cite{eventalign} tries to create an automated mapping solution for this problem using semantics.

At first the data is enriched, Twitter exposes the data using different formats such as JSON \& XML. To make this data more machine readable \cite{eventalign} enrich the data with sematics, after this process the data can then  be used as linked data which is more machine readable. The full enrichment process is described in section 2 of \cite{eventalign}.

%TODO misschien wat meer info over het semantic web toevoegen.
After this process \cite{eventalign} uses multiple techniques to align a tweet with the respective event. The first technique used by \cite{eventalign} is Feature Extraction. Three featuresets are used:
\begin{itemize}
%TODO uitwerken
  \item Immediate Resource Leaves: Use only the URI which are most indicative of an event. This feature only uses literals and resources within the URI of an event. 
  \item 1-Step Resource Leaves: Use all literals and resources within 1 step from the URI of an event. 
  \item DBPedia Concepts: Also uses all the concepts from DBPedia 
\end{itemize}
After this a Feature Vector is composed, for tweets all stop words are removed from the bag of words a tweet is represented as. Furthermore all stopwords are removed and everything is normalized to lowercase. \cite{eventalign} Then a labeling function is used to label the event. 
\subsubsection{Usecase: Earthquake Detection}
In \cite{earthq} a usecase for Event Identification is presented in the form of a earthquake detection system. \cite{earthq} considers twitter users as sensors. If enough sensors are activated (Tweets sent by a user are classified as a tweet about an earthquake), it is likely that an earthquake is occurring. 

To distinguish between tweets about Earthquakes a number of features is used. (1) The number of words in a tweet (2) The words in a tweet (3) Word context features. Feature (1) is a surprisingly good way to classify tweets about earthquakes, especially if it is used with a big amount of users. 

With the above feature set as basis, a system that detects earthquakes using twitter was built. After the system detects an earthquake an email to all users is sent. The system was used during 18. august - 2 September. In this time period 10 earthquakes had occurred, all of these earthquakes were detected. Furthermore \cite{earthq}'s system detects an earthquake 10 minutes after happening and sends a message within a minute after detection. 

\newpage

\section{Conclusion}

\newpage

\bibliographystyle{plain}
\bibliography{bibdb}


\end{document}








