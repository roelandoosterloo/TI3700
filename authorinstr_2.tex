\documentclass{article}
\usepackage{a4wide}
\usepackage{cite}



%% if your are not using LaTeX2e use instead
%% \documentstyle[bnaic]{article}

%% begin document with title, author and affiliations

\title{Seminarium TI 3700\\ Data Analysis on Twitter: An Overview (working title)}
\author{A. Hambenne  \and
    O. Maas \and
    R. Oosterloo}
\date{}

\pagestyle{empty}

\begin{document}
\maketitle
\thispagestyle{empty}

\begin{abstract}
This is the abstract of my paper.
This is the abstract of my paper.
This is the abstract of my paper.
This is the abstract of my paper.
This is the abstract of my paper.
This is the abstract of my paper.
This is the abstract of my paper.
This is the abstract of my paper.
This is the abstract of my paper.
This is the abstract of my paper.
This is the abstract of my paper.
This is the abstract of my paper.
\end{abstract}


\section{Introduction}


\section{Data}

\subsection{Twitter API}

\section{Single tweet analysis}
\subsection{Data enrichment}
\subsection{Sentiment analysis}
\subsection{Concept extraction}
\subsection{Metadata analysis (opt)}

\section{Tweet stream analysis}
\subsection{Statistics}
\subsection{User Modelling}
\subsection{User Classification}


\subsection{Tag based analysis}
\subsection{Event identification}

Event identification is the problem of identifying events(e.g. concerts, festivals \& incidents) in a twitter-stream. The events identified range from Worldwide to local events. Twitter-messages related to an event can be a useful source of information as they often spread news earlier than the 'old' media (e.g News Papers \& News Sites)

To identify an event using an algorithm, a definition of an event is needed. \cite{eventident} defines an event as: "a real world occurence $e$ with (1) an associated Time Period $T_e$ and (2) a time-ordered stream of Twitter messages $M_e$, of substantial volume, discussing the
occurrence and published during time $T_e$."

\subsubsection{Event Identification Issues}
In this section we will discuss several issues regarding Event Identification and the solutions associated with them.
\\\\ 
\textbf{Separation of Event and Non-event Content}\\
Because not all tweets on twitter are event content, a seperation of Event and Non-event content is done prior of identifying events. To achieve this separation, \cite{eventident} uses an incremental clustering algorithm(ICA). ICA doesn't require an number of clusters before the execution of the algorithm. This is an useful feature in the constantly changing twitter-environment.

After the clustering a number of features is used to distinguish between event and non-event clusters. The following features are used by \cite{eventident} to identify an event cluster. 
\begin{itemize}
	%TODO Temporal en Twittercentric uitwerken.
  \item Temporal:
  \item Social: The interactions between users(e.g. retweets, replies \& mentions) during events could differ from other activities. 
  \item Topical: Tweets about events tend to center more around one central topic. Furthermore messages in event clusters are more likely to share one or more event-specific key-terms.  
  \item Twitter-centric: 
\end{itemize}
%TODO Weka toolkit uitzoeken
With this list of features two event-classifiers are trained using two different techniques. The first classifier is trained using the "WEKA tookit", in the second classifier the  Naive-Bayes technique is used. According to \cite{eventident} the first classifier outperforms the second classifier in the separation of event and non-event clusters. 
\\\\
\textbf{Aligning Tweets with an Event} \\
On average thousands of tweets are sent every minute using Twitter(bron) . One of the main challenges in Event Identification is the alignment of tweets with their respective event. \cite{eventalign} tries to create an automated mapping solution for this problem using semantics.

At first the data is enriched, Twitter exposes the data using different formats such as JSON \& XML. To make this data more machine readable \cite{eventalign} enrich the data with sematics, after this process the data can then  be used as linked data which is more machine readable. The full enrichment process is described in section 2 of \cite{eventalign}.

%TODO misschien wat meer info over het semantic web toevoegen.
After this process \cite{eventalign} uses multiple techniques to align a tweet with the respective event. The first technique used by \cite{eventalign} is Feature Extraction. Three featuresets are used:
\begin{itemize}
%TODO uitwerken
  \item Immediate Resource Leaves: TODO
  \item 1-Step Resource Leaves 
  \item DBPedia Concepts 
\end{itemize}
%TODO afmaken

\subsubsection{Usecase: Earthquake Detection}
In \cite{earthq} a usecase for Event Identification is presented in the form of a earthquake detection system. \cite{earthq} considers twitter users as sensors. If enough sensors are activated (Tweets sent by a user are classified as a tweet about an earthquake), it is likely that an earthquake is occurring. 

To distinguish between tweets about Earthquakes a number of features is used. (1) The number of words in a tweet (2) The words in a tweet (3) Word context features. Feature (1) is a surprisingly good way to classify tweets about earthquakes, especially if it is used with a big amount of users. 

With the above feature set as basis, a system that detects earthquakes using twitter was built. After the system detects an earthquake an email to all users is sent. The system was used during 18. august - 2 September. In this time period 10 earthquakes had occurred, all of these earthquakes were detected. Furthermore \cite{earthq}'s system detects an earthquake 10 minutes after happening and sends a message within a minute after detection. 
\section{Conclusion}

\bibliographystyle{plain}
\bibliography{bibdb}


\end{document}








