\documentclass{article}
\usepackage{a4wide}
\usepackage{cite}



%% if your are not using LaTeX2e use instead
%% \documentstyle[bnaic]{article}

%% begin document with title, author and affiliations

\title{Seminarium IN3130\\ Title goes here}
\author{A. Hambenne  \and
    O. Maas \and
    R. Oosterloo}
\date{}

\pagestyle{empty}

\begin{document}
\maketitle
\thispagestyle{empty}

\begin{abstract}
This is the abstract of my paper.
This is the abstract of my paper.
This is the abstract of my paper.
This is the abstract of my paper.
This is the abstract of my paper.
This is the abstract of my paper.
This is the abstract of my paper.
This is the abstract of my paper.
This is the abstract of my paper.
This is the abstract of my paper.
This is the abstract of my paper.
This is the abstract of my paper.
\end{abstract}


\section{Introduction}


\section{Data}

\subsection{Twitter API}

\section{Single tweet analysis}
\subsection{Data enrichment}
\subsection{Sentiment analysis}
\subsection{Concept extraction}
\subsection{Metadata analysis (opt)}

\section{Tweet stream analysis}
\subsection{Statistics}
\subsection{user modeling}
\subsection{Classification}
\subsection{Tag based analysis}
\subsection{Event identification}
\subsubsection{What is Event Identification}
Event identification is the problem of identifying events(e.g. concerts, festivals \& incidents) in a twitter-stream. The events identified range from Worldwide to local events. Twitter-messages related to an event can be a useful source of information spreading information earlier than 'old' media.

To identify an event, a definition of an event is needed. \cite{eventident} defines an event as: "a real world occurence $e$ with (1) an associated Time Period $T_e$ and (2) a time-ordered stream of Twitter messages $M_e$, of substantial volume, discussing the
occurrence and published during time $T_e$."

\subsubsection{Event Identification Issues}

\textbf{Aligning Tweets with an Event} \\
On average thousands of tweets are sent using Twitter every minute. One of the main challenges in Event Identification is the alignment of tweets with their respective event. 
\\\\
\textbf{Separation of Event and Non-event Content}\\
Because not all tweets on twitter are event content, a seperation of Event and Non-event content is done prior of identifying events. To achieve this separation, \cite{eventident} uses an incremental clustering algorithm(ICA). ICA doesn't require an number of clusters before the execution of the algorithm. This is an useful feature in the constantly changing twitter-environment.

After the clustering a number of features is used to distinguish between event and non-event clusters. The following features are used by \cite{eventident} to identify an event cluster. 
\begin{itemize}
  \item Temporal: TODO
  \item Social: The interactions between users(e.g. retweets, replies \& mentions) during events could differ from other activities. 
  \item Topical: Tweets about events tend to center more around one central topic. Furthermore messages in event clusters are more likely to share one or more event-specific key-terms.  
  \item Twitter-centric: TODO
\end{itemize}
With this list of features a two event classifier is trained using two different techniques. The first classifier is trained using the "WEKA tookit"(TODO), in the second classifier the  Naive-Bayes technique is used. According to \cite{eventident} the first classifier outperforms the second classifier in the separation of event and non-event clusters. 
\section{Conclusion}

\bibliographystyle{plain}
\bibliography{bibdb}


\end{document}








