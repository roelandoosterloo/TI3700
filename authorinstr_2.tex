\documentclass{article}
\usepackage{a4wide}


%% if your are not using LaTeX2e use instead
%% \documentstyle[bnaic]{article}

%% begin document with title, author and affiliations

\title{Seminarium IN3130\\ Title goes here}
\author{A. Hambenne  \and
    O. Maas \and
    R. Oosterloo}
\date{}

\pagestyle{empty}

\begin{document}
\maketitle
\thispagestyle{empty}

\begin{abstract}
This is the abstract of my paper.
This is the abstract of my paper.
This is the abstract of my paper.
This is the abstract of my paper.
This is the abstract of my paper.
This is the abstract of my paper.
This is the abstract of my paper.
This is the abstract of my paper.
This is the abstract of my paper.
This is the abstract of my paper.
This is the abstract of my paper.
This is the abstract of my paper.
\end{abstract}


\section{Introduction}


\section{Data}

\subsection{Twitter API}

\section{Single tweet analysis}
In this section we will discuss the different types of analysis that can be done on one single tweet. This usually is also the basis for a analysis of a larger amount of tweets or even on a stream of tweets. 
\subsection{Data enrichment}
At the very basis of a tweet is data enrichment. Because the length of a tweet is restricted to 140 characters the users of Twitter use language in creative ways that sometimes make it, even for humans, hard to understand what a tweet is about. To get a better understanding of the meaning of a tweet, a context is needed. For humans this is done by linking the information from the tweet to earlier acquired knowledge.\\
Some tweets contain an URL, this makes it possible for a computer to create a context from the linked resource. This works 
\subsection{Sentiment analysis}
\subsection{Concept extraction}
\subsection{Metadata analysis (opt)}

\section{Tweet stream analysis}
\subsection{Statistics}
\subsection{user modeling}
\subsection{Classification}
\subsection{Tag based analysis}
\subsection{Event identification}

\section{Conclusion}

\bibliographystyle{plain}
\bibliography{mybibfile}

\end{document}








